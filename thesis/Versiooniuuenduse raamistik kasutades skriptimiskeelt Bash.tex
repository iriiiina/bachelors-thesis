\documentclass[12pt]{report}
\usepackage[utf8]{inputenc}
\usepackage{nameref}
\usepackage{fancyhdr}
\usepackage{hyperref}
\usepackage{caption}
\usepackage{csquotes}
\renewcommand{\contentsname}{Sisukord}
\renewcommand{\bibname}{Kasutatud materjalid}
\renewcommand{\thesection}{\arabic{section}}
% Page numbers to the right
\pagestyle{fancy}
\fancyhead{}
\fancyfoot{}
\fancyfoot[R]{\thepage}
\renewcommand{\headrulewidth}{0pt}
% Code formatting
\newcommand{\code}[1]{\texttt{#1}}

\title{Versiooniuuenduse raamistik kasutades skriptimiskeelt Bash}
\author{Irina Ivanova}
\date{January 2016}

\begin{document}
  \begin{titlepage}
    \begin{center}
      TARTU ÜLIKOOL\\
      MATEMAATIKA-INFORMAATIKATEADUSKOND\\
      Arvutiteaduse instituut\\
      Infotehnoloogia eriala
    \end{center}
      
    \vspace{5cm}
    
    \begin{center}  
      {\Large Irina Ivanova}
    \end{center}
    \begin{center}      
      {\huge Versiooniuuenduse raamistik}
    \end{center}
    \begin{center}
      {\huge kasutades skriptimiskeelt Bash}
    \end{center}
    \begin{center}
      {\large Bakalaureusetöö (6 EAP)}
    \end{center}
      
    \vspace{4cm}
    \hspace{4.5cm}
    Juhendajad: dotsent Helle Hein
    
    \hspace{7cm}
    Polina Morozova, MSc (Nortal AS)
      
    \vspace{2.5cm}
    \begin{center}
      TARTU 2016
    \end{center}
  \end{titlepage}

  \newpage
  
  \noindent{\textbf{Versiooniuuenduse raamistik kasutades skriptimiskeelt Bash}}\\
  
  \noindent{\textbf{Lühikokkuvõte:}}\\
  \vspace{2cm}
  
  \noindent{\textbf{Võtmesõnad:}}\\
  Bash skript, versiooni uuendus, Apache Tomcat, Java
  \vspace{2cm}

  \noindent{\textbf{Thesis title}}\\

  \noindent{\textbf{Abstract:}}\\
  \vspace{2cm}

  \noindent{\textbf{Keywords:}}\\
  Bash script, version update, Apache Tomcat, Java
  
  \newpage
 
  \tableofcontents
  \fancypagestyle{plain}{%
    \renewcommand{\headrulewidth}{0pt}%
    \fancyhf{}%
    \fancyfoot[R]{\thepage}%
}
 
  \newpage
  
  % Sissejuhatus
  \section*{Sissejuhatus}
  \label{sissejuhatus}
  \addcontentsline{toc}{section}{\nameref{sissejuhatus}}
  
  Käesolevas lõputöös lahendatakse rakenduslikku probleemi toimivas tarkvaraprojektis. Probleem seisneb selles, et projekti olemuse tõttu toote versiooni uuendamise protsess võtab kaua aega ja eeldab palju manuaalset sekkumist. See raiskab resursse ja suurendab inimliku vea tekkimise tõenäosust. Töö eesmärk on automatiseerida versiooniuuenduse protsessi konkreetses projektis, selleks, et lihtsustada antud protsessi, et säästa aega ja vähendada inimlike vigade arvu.
  
  Töö koosneb viiest osast. Esimeses osas püstitatakse probleem: kirjeldatakse projekti arhitektuuri, manuaalse uuendamise protsessi, probleeme, mis vana protsess toob ja nõudeid uuele lahendusele. Peamine väljakutse seisneb selles, et toode koosneb paarkümnest moodulitest (täpne arv sõltub kliendi nõuetest), mida on vaja paigaldada 31 keskkonna. See tähendab, et selleks, et paigaldada toote uut versiooni kõikidele keskkondadele on vaja teha umbes 620 uuendust (mitte arvestades testkeskkondade vaheuuendustega). Lisaks on projektis kasutusel kaks erinevat veebiserverit: Oracle WebLogic 12 ja Apache Tomcat 8. Uuenduse protsessi automatiseerimine peab võtma arvesse kõik need väljakutsed.
  
  Teises osas uuritakse võimalikke lahendusi: kas kasutada turul olemasolevat toodet (töös uuritakse neli toodet: Atlassian Bamboo, Chef, Jenkins ja Ansible Tower) või kirjutada ise tarkvara. Tuuakse välja iga lahenduse eeliseid ja puuduseid ning seletatakse miks langetati otsus Bash skripti kirjutamise kasuks. Valmistoodete peamised puudused on maksmus, võimetus hallata protsesse ja vajadusel parandada vigu, vajadus kulutada palju aega toode tundmiseks ja rakendamiseks.
  
  Kolmandas osas kirjeldatakse valitud lahendust: mis on skripti tööpõhimõtte ja arhitektuur ja kuidas seda saab paigaldada ja kasutada. Skripti põhifunktsionaalsus on mooduli faili alla laadimine, vana versiooni maha võtmine, uue versiooni paigaldamine, kolleegide teavitamine toimunud versiooniuuendusest. Lisaks oluliseks funktsionaalsuseks on erinevad kontrollid, mis vähendavad tõenäosust paigaldada vale versioon ja logimise funktsionaalsus, mis võimaldab kätte saada põhjalikku infot iga uuenduse kohta.
  
  Neljandas osas uuritakse kuidas antud lahendus mõjus projektile: võrreldakse versiooni uuendusele kulunud aega manuaalse ja automatiseeritud protsessi puhul ja lisatakse kolleegide tagasisidet uue raamistiku kasutamisele.
  
  Viimases osas tuuakse välja lahenduse perspektiive: kuidas loodud skripti saab edasi arendada ja kuidas saab seda teistes projektides rakendada.
  
  Töö tulemuseks on töötav, avatud lähtekoodiga, skript Bash keeles ja selle dokumentatsioon.
  
  \newpage
  
  % Probleem
  \section{Probleemi püstitus}
  
  Versioonide uuendamine on tingimata vajalik protsess tarkvara arenduses. Projektis, mida kirjeldatakse käesoleva töö raames, see protsess mängib väga olulist rolli.
  
  Projekti rakendus on kirjutatud Java keeles ja on kasutusel .WAR failid. Lisaks projekt rakendub modulaarsuse süsteemi, mis tähendab, et kogu rakenduse kood on jagatud moodulite kaupa ja iga mooduli kohta on olemas eraldi .WAR fail. Kokku on umbes 20 moodulit (täpne arv sõltub kliendist, sest erinevatel klientidel on erinev moodulite komplekt).
  
  Rakendust kasutavad 10 klienti. 3-l nendes on olemas 3 keskkonda: toode, demo (kus kliendid tutvuvad uue funktsionaalsuse ja parandustega enne toote uuendust) ja test (kus projekti meeskond testib funktsionaalsust enne tarnimist kliendile). 7-l kliendil on olemas 2 keskkonda: toode ja test. See tähendab, et kokku on olemas $3*3+7*2=23$ kliendikeskkonda. Lisaks veel umbes 7 projekti test keskkonda, mida kasutatakse erinevatel faasidel erineva funktsionaalsuse testimise jaoks. Seega kokku tuleb $23+7=30$ keskkonda, mida on vaja regulaarselt uuendada (mitte arvestades lokaalsete ja virtuaalmasinatega, mida iga meeskonnaliige võib vajadusel panna püsti). See tähendab, et selleks, et paigaldada rakenduse uut versiooni igale keskkonnale on vaja teha $30*20=600$ uuendust.
  
  Toodud numbrid näitavad, et versiooni manuaalne uuendus võtab palju aega ja tähelepanu, mille pärast oli otsustatud lihtsustada ja automatiseerida uuenduse protsessi.
  
  \newpage
  \subsubsection{Manuaalne uuenduse protsess}
  
  Ühe mooduli manuaalse uuenduse protsessi sammud on nimetatud allpool tabelis 1 "Manuaalse uuenduse protsessi sammud".
  
  \begin{table}[!htbp] 
  \begin{tabular}{ |c|p{11cm}| }
    \hline
    \textbf{Number} & \textbf{Samm}\\
    \hline
    1 & .WAR faili uue koodiga alla laadimine serveri peale.\\
    \hline
    2 & Kui veebiserveriks on Oracle WebLogic, siis .WAR faili prekompileerimine (Apache Tomcat veebiserveril seda sammu ei ole).\\
    \hline
    3 & Faili ümber nimetamine.\\
    \hline
    4 & Faili paigaldus.\\
    \hline
    5 & Vana faili eemaldamine.\\
    \hline
    6 & Uue versiooni staatuse kontrollimine.\\
    \hline
    7 & JIRA töö uuendamine, mis automaatselt saadab emaile kolleegidele, et versioon on uuendatud (iga mooduli ja keskkonna kohta on olemas eraldi JIRA töö).\\
    \hline
    8 & Alla laaditud faili kustutamine.\\
    \hline
  \end{tabular}
  \caption*{\textit{Tabel 1: Manuaalse uuenduse protsessi sammud}}
  \end{table}
  
  Mitmete moodulite uuendamisel tuleb teha kõik need sammud iga mooduli jaoks. Aeg, mida võtavad need sammud, sõltub moodulist (kuna mõnes moodulis võib olla kaks korda rohkem koodi, kui teises, mis tähendab, et selle alla laadimine ja paigaldamine võtab kaks korda rohkem aega), aga keskmiselt see on 1.5 minutit. Kõikide moodulite uuendamine võib võtta tund aega, kuna lisandub veel aeg, mida inimene raisab õige versiooni numbri leidmiseks, .WAR faili aadressi kopeerimiseks, õige JIRA töö leidmiseks ja avamiseks jne. See tähendab, et inimlike vigade tekkimise tõenäosus on suur (nt alla laadida vale versiooni).
  
  Allpool on toodud Tabel 1 "Nõuded versiooni uuenduse lihtsustamise lahendusele", mis kirjeldab lahendust, millele on mõtet investeerida aega protsessi muutmiseks \--- muul juhul uue protsessi loomine võtab rohkem aega, kui toob kasu.
  
  \begin{table}
  \begin{tabular}{|c|p{11cm}|}
    \hline
    \textbf{Number} & \textbf{Nõue}\\
    \hline
    1 & Uuenduse protsess peab võtma vähem aega.\\
    \hline
    2 & Uuenduse protsess peab nõudma nii vähe käsitsi tegevusi, kui on võimalik.\\
    \hline
    3 & Lahendus peab võimaldama uuendada nii üht moodulit, kuid ka mitu moodulit korraga.\\
    \hline
    4 & Projekt kasutab kaks erinevat veebiserverit: Apache Tomcat 8 ja Oracle WebLogic 12. Lahendus peab töötama sama moodi kõikidel serveritel, sõltumata sellest, mis tarkvara seal on paigaldatud.\\
    \hline
    5 & Mõned meeskonnaliikmed soovivad saada teavitusi toimunud uuenduste kohta, mis tähendab, et lahendusel peab olema teavituste süsteem.\\
    \hline
    6 & Peab olema võimalus vaadata kes ja millal uuendas mingit moodulit mingis keskkonnas, mis tähendab, et peab olema logimise süsteem.\\
    \hline
    7 & Uuendustega tegelevad testijad, seega protsess ei tohi olla liiga tehniline.\\
    \hline
    8 & Uuendustega võivad tegeleda mitu inimest samal ajal, mis tähendab, et lahendusel peab olema lukustamise süsteem, et vältida paralleelselt uuendust samal serveril.\\
    \hline
    9 & Toodang keskkonnad kasutavad 2 serverit, seega lahendus peab oskama uuendada versiooni automaatselt kahel serveril.\\
    \hline
    10 & Lahendus peab olema täielikult projekti kontrolli all, selleks, et selle haldamine, seadistamine ja parandamine oleks võimalik teostada igal ajal.\\
    \hline
    11 & Lahenduse loomine ei tohi võtta palju aega, sest projektil puuduvad selleks inimresurssid.\\
    \hline
    12 & Lahendus ei tohi võtta palju raha.\\
    \hline
  \end{tabular}
  \caption*{\textit{Tabel 2: Nõuded versiooni uuenduse lihtsustamise lahendusele}}
  \end{table}
  
  \newpage
  
  % Võimalikud lahendused
  \section{Võimalikud lahendused}
  
  Uuenduse protsesside automatiseerimiseks ja haldamiseks on olemas palju valmistoodet. Kõige populaarsemad nendest on Atlassian Bamboo \cite{bamboo}, Chef \cite{chef}, Jenkins \cite{jenkins} ja Ansible Tower \cite{ansible}.
  
  Kõikide valmistoodete rakendamisel alati on olemas oma eelised ja puudused.
  
  \textbf{Eelised:}
  \begin{itemize}
    \item Ei ole vaja kulutada aega süsteemi kirjutamisele, mis tegelikult on juba leiutatud.
    \item Kuna rakendus on juba turul ja on populaarne, siis on suur tõenäosus, et see on kvaliteetne ja vastab kasutajate ootustele.
    \item On olemas kogukond tarkvara kasutajatest, kes vajadusel saavad aidata ja jagada enda kogemust.
  \end{itemize}
  
  \textbf{Puudused:}
  \begin{itemize}
    \item Suurem osa valmistoodetest on tasulised (Bamboo ja Jenkins on erandiks) ja hind alati sõltub serverite, keskkondade ja moodulite arvust, mis vaadatud projekti korral on suur.
    \item Valmistoode õppimine, paigaldamine ja seadistamine võtab palju aega, eriti kui see pakkub lai valik funktsionaalsust.
    \item Projekt väga sõltub tootjast \-- vigade olemas olul ei saa olla kindel, et neid parandatakse lähimal ajal või parandatakse üldse; probleemide tekitamisel ei saa olla kindel, et tootja kasutajate tugi saab aidata.
    \item Projekt peab usaldama tootjat \-- usaldama, et valmistoode on turvaline ja stabiilne.
  \end{itemize}
  
  Need on ühtlased eelised ja puudused kõikide valmistoodete kohta.
  
  Allpool on toodud Tabel 3 "Võimalikute lahenduste vastamine nõuetele", kus on määratud millistele nõuetele Tabelist 1 vastavad vaadeldud lahendused.
  
  \begin{table}
    \begin{tabular}{ |l|c|c|c|c|c|c|c|c|c|c|c|c| }
      \hline
      \textbf{Lahendus / Nõude Nr} & \textbf{1} & \textbf{2} & \textbf{3} & \textbf{4} & \textbf{5} & \textbf{6} & \textbf{7} & \textbf{8} & \textbf{9} & \textbf{10} & \textbf{11} & \textbf{12}\\
      \hline
      Atlassian Bamboo & x & x & & x & x & x & x & & & & & x\\
      \hline
      Chef & ? & ? & ? & x & x & x & ? & ? & ? & x & & \\
      \hline
      Jenkins & ? & ? & ? & x & x & x & ? & ? & ? & x & & x\\
      \hline
      Ansible Tower & ? & ? & ? & x & x & x & ? & ? & ? & & & \\
      \hline
      Bash skript & x & x & x & x & x & x & x & x & x & x & x & x\\
      \hline
    \end{tabular}
    \caption*{\textit{Tabel 3: Võimalikute lahenduste vastamine nõuetele}}
  \end{table}

  
  \subsubsection{Atlassian Bamboo}
  
  \textbf{Eelised:}
  \begin{itemize}
    \item Tasuta, sest projekt juba kasutab teised Atlassian tooted.
    \item Kuna projektis kasutatakse Atlassian toodeid (JIRA, Confluence, Fisheye), siis Bamboo väga hästi integreerub nendega. On võimalik mudagvalt teha seoseid rakenduse koodi, muudatuste kirjelduse ja uuenduse vahel.
    \item Antud toodet kasutavad teised projektid firmas, mis annab võimalust kaasata inimesi, kellel on olemas teadmised ja kogemus Bamboo seadistamise kohta.
    \item Mugav kasutamine veebilehitseja kaudu.
  \end{itemize}
  
  \textbf{Puudused:}
  \begin{itemize}
    \item Seadistamine toimub moodulite kaupa (igal moodulil on oma Bamboo projekt) ja ühe mooduli seadistamine võtab palju aega. Firma teises projektis see aeg oli 1.5 kuud, mis antud projekti jaoks tähendaks ~20 * 1.5 = ~30 kuud.
    \item Üks uuendus võib võtta palju aega, sest Bamboo tööpõhimõte eeldab kolmanda serveri kasutamist uuenduse tööplaani teostamiseks ja selliste serverite arv on piiratud. See tähendab, et kui korraga soovitakse uuendada mitu keskkonda, siis võib tekkida ootusjärjekord.
  \end{itemize}
  
  \subsubsection{Chef}
  
  \textbf{Eelised:}
  \begin{itemize}
    \item 
  \end{itemize}
  
  \textbf{Puudused:}
  \begin{itemize}
    \item 
  \end{itemize}
  
  \subsubsection{Jenkins}
  
  \textbf{Eelised:}
  \begin{itemize}
    \item Tasuta.
    \item Avalik kood, mis tähendab, et projekt ei sõltu tootjast.
    \item Firmas on olemas spetsialistid, kes on kasutanud Jenkins-t ja saavad jagada teadmisi ja kogemust.
  \end{itemize}
  
  \textbf{Puudused:}
  \begin{itemize}
    \item 
  \end{itemize}
  
  \subsubsection{Ansible Tower}
  
  \textbf{Eelised:}
  \begin{itemize}
    \item 
  \end{itemize}
  
  \textbf{Puudused:}
  \begin{itemize}
    \item
  \end{itemize}

  \subsubsection{Bash skriptimine \cite{bash}}
  
  \textbf{Eelised:}
  \begin{itemize}
    \item Tasuta.
    \item Lahendus on täiesti projekti kontrolli all, seega saab olla kiiresti parandatud ja vajadusel muudetud või täiendatud.
    \item Lahendust on võimalik areneda osade kaupa: alguses teha kõige lihtsamat funktsionaalsust, hiljem järk-järgult täiendada seda uue ja keerulise funktsionaalsusega. See võimaldab kasutada ressurse väikeste osadena ja samal ajal töötada juba natukene parema süsteemiga.
    \item Lahendus on väga paindlik: on võimalik areneda just seda funktsionaalsust, mis on vajalik projekti jaoks ja lisada seadistamise võimalust, et sama lahendust saaks kasutada erinevate serverite ja keskkondade korral.
    \item Lahendus ei nõua projekti arhitektuuri ja protsesside muutmist: arendajate jaoks uue versiooni tekitamise protsess jääb samaks.
  \end{itemize}
  
  \textbf{Puudused:}
  \begin{itemize}
    \item Täieliku lõpplahenduse kirjutamine võtab palju aega.
    \item Lahendus ei ole integreeritud teiste Atlassian toodetega, seega ei ole seotud koodi muudatuste (Fisheye commit) ja tööülesannetega (JIRA issue).
  \end{itemize}
  
  \newpage
  
  % Lahendus
  \section{Lahendus}
  
  Versiooni uuenduse protsessi lihtsustamiseks oli valitud Bash skripti kirjutamine.
  
  \subsection{Skripti arhitektuur ja tööpõhimõte}
  
  \subsection{Skripti paigaldus}
  
  \subsection{Skripti kasutus}
  
  \subsection{Dokumentatsioon}
  
  \newpage
  
  % Mõju projektile
  \section{Mõju projektile}
  
  Automaatne versiooni uuenduse raamistik täielikult asendas vana käsitsi protsessi. Selles on olemas nii eelised, kuid ka oma puudused.\\
  
  Kõige oluline eelis on see, et skript täidab kõik nõuded tabelist 2 "Nõuded versiooni uuenduse lihtsustamise lahendusele". Kõige märkavamad nendest on nõuded 1 ja 2 \--- protsessi kiirus ja mugavus.\\
  
  Kiirus on ainuke tunnus, mida saab objektiivselt mõõta. Vana ja uue protsesside kiirused olid mõõdetud kasutades \textit{UNIX} käsurea funktsiooni \code{time} \cite{time}. Oli valitud 4 protsessi: 3 üksikute kõige populaarsemate moodulite uuendamine ja kõikide moodulite uuendamine ühes keskkonnas. Iga protsess oli tehtud vana käsitsi viisi ja uue automaatse skriptiga 3 korda ja võrdluseks on võetud katsetuste keskmised väärtused. Tulemused on toodud tabelis 4 "Vana ja uue protsesside kiirused".
  
  \begin{table}[!htbp]
  \begin{tabular}{| l | c | c |}
    \hline
    Moodul & Aut. uuendus (sek) & Käsitsi uuendus (sek)\\
    \hline
    admin & 27.245 & 80.45\\
    treatment & 61.101 & 112.71\\
    reception & 38.538 & 100.19\\
    kõik (21) & & \\
    \hline
  \end{tabular}
  \caption*{\textit{Tabel 4: Vana ja uue protsesside kiirused}}
  \end{table}
  
  Tegelikkuses kokkuhoidud aeg on veel suurem, sest vanas protsessis uuendaja tähelepanu oli hõivatud terve protsessi jooksul. Uue skripti kasutades testijatel on olemas võimalus tegeleda paralleelselt teiste ülesannetega ja mitte jälgida skripti tööd (see ise annab teada, kui uuendus on lõppenud).\\
  
  Kõige oluline puudus on see, et testijad enam ei tunne versiooni uuenduse protsessi tausta ja vigade tekitamise juhul (kas serveri, veebiserveri või skripti pärast) ei oska neid lahendada. See on üldine automatiseerimise probleem, millega tuleb lepida mugavuse vahetuseks.\\
  
   Käesoleva töö raames oli küsitud tagasiside uue raamistiku kohta meeskonnaliikmete käest, kes on seda kasutanud. Allpool on toodud mõned väljavõted, terved tekstid on kättesaadavad lisas 5.5.
   
   \begin{displayquote}
   ``I also like that scalability has been considered when writing the script + many optional features have been implemented which can be used if one desires to do so.'' (Martin Rakver, Hosting Services and Application Manager)
   \end{displayquote}
   
   \begin{displayquote}
   ``I personally think that version-update script has significantly improved the speed of version update.'' (Kalle Jagula, QA Specialist)
   \end{displayquote}
   
   \begin{displayquote}
    ``Version updater script has standardized the way our environments are updated, reducing learning curve for new people performing the task. Downside is that people are unaware, what happens in the background and can't handle simple errors in the process anymore.'' (Tanel Käär, System Architect)
    \end{displayquote}
    
    \begin{displayquote}
    ``The scripts allowed me to conveniently perform the updates while letting me concentrate on solving the primary problems and not leaving the environment containing all necessary information.'' (Klaus-Eduard Runnel, Senior Programmer)
    \end{displayquote}
    
    \begin{displayquote}
    ``Samuti tooks välja, et uuendamise skriptid on väga hästi dokumenteeritud ja põhjalikult kirja pandud kuidas toimivad.'' (Helina Ziugand, QA Specialist)
    \end{displayquote}

  \newpage
  
  % Lahenduse perspektiivid
  \section{Lahenduse perspektiivid}  
  
  \newpage
  
  % Kokkuvõte
  \section*{Kokkuvõte}
  \label{kokkuvote}
  \addcontentsline{toc}{section}{\nameref{kokkuvote}}

Kokkuvõttes tuuakse selgelt välja töö põhilised saavutused. Ei tohi sisse tuua uusi väiteid, põhjendusi ja analüüse, mida töö põhiosas ei ole käsitletud. Mõistlik on viidata eesmärgile ja sõnastada kokkuvõte nii, et on näha eesmärgi saavutamine ning ka autori panus.\\

Kokkuvõttes võib lühidalt esile tuua töö edasiarendamise võimalikke teid ja perspektiive.\\

Kokkuvõtte tekstis on soovitav kasutada lihtmineviku umbisikulist tegumoodi, näiteks "Töös toodi välja …, kirjeldati …, leiti lahendus … ", aga ka olevik on vastuvõetav.

  \newpage
  
  % Kasutatud materjalid 
  \begin{thebibliography}{9}
    \label{kasutatud-materjalid}
    \addcontentsline{toc}{section}{\nameref{kasutatud-materjalid}}
  
    \bibitem{bash}
    \url{https://www.gnu.org/software/bash/manual/bashref.html}
    
    \bibitem{tomcat}
    \url{https://tomcat.apache.org/tomcat-8.0-doc/manager-howto.html}
  
    \bibitem{cron}
    \url{https://help.ubuntu.com/community/CronHowto}
  
    \bibitem{hg}
    \url{https://www.mercurial-scm.org/guide}
  
    \bibitem{centos}
    \url{https://wiki.centos.org/FrontPage}
  
    \bibitem{bamboo}
    \url{https://www.atlassian.com/software/bamboo}
  
    \bibitem{chef}
    \url{https://www.chef.io}
  
    \bibitem{jenkins}
    \url{https://jenkins-ci.org}
  
    \bibitem{ansible}
    \url{https://www.ansible.com}
    
    \bibitem{time}
    \url{http://pubs.opengroup.org/onlinepubs/9699919799/utilities/time.html}
 
  \end{thebibliography}

  \newpage
  
  % Lisad
  \section*{Lisad}
  \label{lisad}
  \addcontentsline{toc}{section}{\nameref{lisad}}
  
  \subsection{Terminid}
  
  \textbf{Raamistik} \--- \\
  \textbf{Rakendus} \--- \\
  \textbf{Server} \--- \\
  \textbf{Keskkond} \--- \\
  \textbf{Moodul} \--- \\
  \textbf{Versioon} \--- \\
  \textbf{Modulaarne süsteem} \--- \\
  \textbf{Veebiserver} \---
  
  \subsection{Skripti lähtekood}
  
  Bash skripti kood asub avalikus GitHub reposetooriumis: \url{https://github.com/iriiiina/version-updater}
  
  \subsection{Skripti avalik juhend}
  
  Bash skripti avalik juhend asub avalikus GitBook reposetoorimus: \url{https://iriiiina.gitbooks.io/version-updater-manual/content/}
  
  \subsection{Skripti projekti juhend}
  
  Bash skripti projekti sisene juhend: \url{https://github.com/iriiiina/bachelors-thesis/blob/master/manual/Confluence%20Manual.pdf}
  
  \subsection{Kolleegide tagasiside}
  
  \subsubsection{Martin Rakver, Hosting Services and Application Manager}
  
  I have not been using the script very much, because I rarely update modules nowadays. But what I can say is that it is definately in the right place – automating activities that testers would need do to on daily basis each time when deploying a new module. Come to think about it, there are actually quite many activites related to new module deployment – would be interesting to compare the time it takes to deploy a module manually vs using script + calculate about how much time we as a team are saving daily, montly, yearly (and get to do more important things with that time).
I also like that scalability has been considered when writing the script + many optional features have been implemented which can be used if one desires to do so.

  \subsubsection{Kalle Jagula, QA Specialist}
  
  I personally think that version-update script has significantly improved the speed of version update. Especially with the settings for bulk-update, which improves the speed of modules' version update and manageability of test/demo environments.
  
  \subsubsection{Tanel Käär, System Architect}
  
  Version updater script has standardized the way our environments are updated, reducing learning curve for new people performing the task. Downside is that people are unaware, what happens in the background and can't handle simple errors in the process anymore.
  
  \subsubsection{Klaus-Eduard Runnel, Senior Programmer}
  
  I have mostly used the scripts to perform quick module and dependency updates in specific test environments when deploying quick fixes or debugging problems in the modules I'm responsible for. Deployment of artifacts is secondary task in such scenarios. The scripts allowed me to conveniently perform the updates while letting me concentrate on solving the primary problems and not leaving the environment containing all necessary information.

In some cases it was necessary to deploy custom-built experimental (unversioned) war-files. In these cases the scripts could not be used.
They would have been useful though if such experiments would have been commited to feature branches and published as feature branch artifacts. (As of today, we are better prepared for such circumstances.)

The scripts were also useful to deploy predefined sets of modules to temporary test environments living on reusable virtual machine images. It is important to note that a module update triggers launching a set of associated sql scripts to bring a database schema up to date. In that way the version update scripts let us automate most of a process of setting up updated environment from virtual machine images.

It would be helpful if the scripts were adapted to perform changes on remote machines and to perform changes on multiple environments simultaneously.

  \subsubsection{Helina Ziugand, QA Specialist}
  
  Uuendamise skriptid on nii projektile kui ka minu igapäevasele tööle väga positiivselt mõjunud. 
Kui algselt oli uuendamise jaoks vaja mitu sammu käsitsi teha, siis nüüd käib see automaatselt ja kiiresti. Võiks öelda, et üsna tüütu oli alguses kogu see uuendamise protsess - pidevalt pidi jälgima, kas vaja käsureale kirjutada järgmist sammu, hiljem JIRAs vana versiooni numbrit muuta ja Weblogicust vana versioon kustutada. Praegu on väga mugav ühe käsklusega uuendamise skript tööle panna, mis korraga kõik vajalikud sammud ära teeb ja samal ajal kui skript jookseb ise teiste tegevustega jätkata. Väga selgelt erinevate värvidega on välja tootud error, warningu ja success teated, mis on lihtsasti märgatavad ja hästi loetavad. 

Kui vaja on rohkem kui ühte moodulit uuendada, siis väga hea lahendus on selleks mitme moodulise uuendamise skript. Kui varem oli vaja keskkonda uue tsükli peale uuendada, siis võttis see ebamugavalt kaua aega ja vabatahtlikult kõige parema meelega seda ette ei tahtnud võtta. Nüüd on vaja ainult uuendatavad versioonid tekstifaili kirja panna ja ühe käsklusega skript käima tõmmata. 

Samuti tooks välja, et uuendamise skriptid on väga hästi dokumenteeritud ja põhjalikult kirja pandud kuidas toimivad. Mul on väga hea meel, et versioonide uuendamine on skriptide abil nii lihtsaks, mugavaks ja meeldivaks tegevuseks saanud.


\end{document}