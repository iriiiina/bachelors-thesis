\documentclass[12pt]{report}
\usepackage[utf8]{inputenc}
\usepackage{nameref}
\usepackage{fancyhdr}
\renewcommand{\rmdefault}{ptm}
\renewcommand{\contentsname}{Sisukord}
\renewcommand{\thesection}{\arabic{section}}
% Page numbers to the right
\pagestyle{fancy}
\fancyhead{}
\fancyfoot{}
\fancyfoot[R]{\thepage}
\renewcommand{\headrulewidth}{0pt}

\title{Versiooniuuenduse raamistik kasutades skriptimiskeelt Bash}
\author{Irina Ivanova}
\date{January 2016}

\begin{document}
  \begin{titlepage}
    \begin{center}
      TARTU ÜLIKOOL\\
      Matemaatika ja statistika insituut\\
      Infotehnoloogia eriala
    \end{center}
      
    \vspace{5cm}
    
    \begin{center}  
      {\Large Irina Ivanova}
    \end{center}
    \begin{center}      
      {\huge Versiooniuuenduse raamistik}
    \end{center}
    \begin{center}
      {\huge kasutades skriptimiskeelt Bash}
    \end{center}
    \begin{center}
      {\large Bakalaureusetöö (6 EAP)}
    \end{center}
      
    \vspace{4cm}
    \hspace{6cm}
    Juhendajad: Polina Morozova
    
    \hspace{8.15cm}
    Helle Hein
      
    \vspace{1.5cm}
    \begin{center}
      TARTU 2016
    \end{center}
  \end{titlepage}

  \newpage
  
  Versiooniuuenduse raamistik kasutades skriptimiskeelt Bash\\
  
  Lühikokkuvõte:\\
  \vspace{2cm}

  Võtmesõnad:\\
  \vspace{2cm}

  Thesis title\\

  Abstract:\\
  \vspace{2cm}

  Keywords:
  
  \newpage
 
  \tableofcontents
  \fancypagestyle{plain}{%
    \renewcommand{\headrulewidth}{0pt}%
    \fancyhf{}%
    \fancyfoot[R]{\thepage}%
}
 
  \newpage
  
  \section*{Sissejuhatus}
  \label{sissejuhatus}
  \addcontentsline{toc}{section}{\nameref{sissejuhatus}}
  
  \begin{itemize}
    \item teema valiku põhjendus (teema aktuaalsus ja uudsus)
    \item töö eesmärk
    \item  uuritav probleem (vajadusel püstitatud hüpotees(id), uurimisküsimus(ed), uuringu objekt)
    \item töö struktuuri kirjeldus peatükkide kaupa. Samuti tutvustakse lühidalt lisasid, sh kaasapandud
materjalide sisu.
  \end{itemize}
  
  \newpage
  
  \section{Probleemi kirjeldus}
  
  \subsection{Probleem}
  
  \subsection{Serverite arhitektuur}
  
  \begin{tabular}{|l|c|c|c|c|c|c|c|}
    \hline
    Environment & Count & Web Server & Auth & Silent & Restart & Batch & Multi-Servers\\
    \hline
    Test-TC & 10 & Tomcat 8 & Y & N & N & Y & N\\
    Test2-TC & 2 & Tomcat 8 & Y & Y & Y & N & Y\\
    Live-TC & 7 & Tomcat 8 & Y & Y & N & Y & N\\
    Live2-TC & 1 & Tomcat 8 & Y & Y & Y & N & Y\\
    Local & N* & Tomcat 8 & N & N & N & Y & N\\
    Test-WL & 4 & WebLogic 12 & Y & N & N & Y & N\\
    \hline
  \end{tabular}

  
  \subsection{Nõuded lahendusele}
  
  \newpage
  
  \section{Alternatiivid}
  
  \newpage
  
  \section{Lahendus}
  
  \newpage
  
  \section*{Kokkuvõte}
  \label{kokkuvote}
  \addcontentsline{toc}{section}{\nameref{kokkuvote}}

Kokkuvõttes tuuakse selgelt välja töö põhilised saavutused. Ei tohi sisse tuua uusi väiteid, põhjendusi ja analüüse, mida töö põhiosas ei ole käsitletud. Mõistlik on viidata eesmärgile ja sõnastada kokkuvõte nii, et on näha eesmärgi saavutamine ning ka autori panus.\\

Kokkuvõttes võib lühidalt esile tuua töö edasiarendamise võimalikke teid ja perspektiive.\\

Kokkuvõtte tekstis on soovitav kasutada lihtmineviku umbisikulist tegumoodi, näiteks "Töös toodi välja …, kirjeldati …, leiti lahendus … ", aga ka olevik on vastuvõetav.

  \newpage
  
  \section*{Kasutatud materjalid}
  \label{kasutatud-materjalid}
  \addcontentsline{toc}{section}{\nameref{kasutatud-materjalid}}

  \newpage
  
  \section*{Lisad}
  \label{lisad}
  \addcontentsline{toc}{section}{\nameref{lisad}}

\end{document}