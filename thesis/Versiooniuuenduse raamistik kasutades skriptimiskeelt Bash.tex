\documentclass[12pt]{report}
\usepackage[utf8]{inputenc}
\usepackage{nameref}
\usepackage{fancyhdr}
\usepackage{hyperref} 
\renewcommand{\contentsname}{Sisukord}
\renewcommand{\bibname}{Kasutatud materjalid}
\renewcommand{\thesection}{\arabic{section}}
% Page numbers to the right
\pagestyle{fancy}
\fancyhead{}
\fancyfoot{}
\fancyfoot[R]{\thepage}
\renewcommand{\headrulewidth}{0pt}

\title{Versiooniuuenduse raamistik kasutades skriptimiskeelt Bash}
\author{Irina Ivanova}
\date{January 2016}

\begin{document}
  \begin{titlepage}
    \begin{center}
      TARTU ÜLIKOOL\\
      MATEMAATIKA-INFORMAATIKATEADUSKOND\\
      Arvutiteaduse instituut\\
      Infotehnoloogia eriala
    \end{center}
      
    \vspace{5cm}
    
    \begin{center}  
      {\Large Irina Ivanova}
    \end{center}
    \begin{center}      
      {\huge Versiooniuuenduse raamistik}
    \end{center}
    \begin{center}
      {\huge kasutades skriptimiskeelt Bash}
    \end{center}
    \begin{center}
      {\large Bakalaureusetöö (6 EAP)}
    \end{center}
      
    \vspace{4cm}
    \hspace{4.5cm}
    Juhendajad: dotsent Helle Hein
    
    \hspace{7cm}
    Polina Morozova, MSc (Nortal AS)
      
    \vspace{2.5cm}
    \begin{center}
      TARTU 2016
    \end{center}
  \end{titlepage}

  \newpage
  
  \noindent{\textbf{Versiooniuuenduse raamistik kasutades skriptimiskeelt Bash}}\\
  
  \noindent{\textbf{Lühikokkuvõte:}}\\
  \vspace{2cm}
  
  \noindent{\textbf{Võtmesõnad:}}\\
  Bash skript, versiooni uuendus
  \vspace{2cm}

  \noindent{\textbf{Thesis title}}\\

  \noindent{\textbf{Abstract:}}\\
  \vspace{2cm}

  \noindent{\textbf{Keywords:}}\\
  Bash script, version update
  
  \newpage
 
  \tableofcontents
  \fancypagestyle{plain}{%
    \renewcommand{\headrulewidth}{0pt}%
    \fancyhf{}%
    \fancyfoot[R]{\thepage}%
}
 
  \newpage
  
  % Sissejuhatus
  \section*{Sissejuhatus}
  \label{sissejuhatus}
  \addcontentsline{toc}{section}{\nameref{sissejuhatus}}
  
  \begin{itemize}
    \item teema valiku põhjendus (teema aktuaalsus ja uudsus)
    \item töö eesmärk
    \item  uuritav probleem (vajadusel püstitatud hüpotees(id), uurimisküsimus(ed), uuringu objekt)
    \item töö struktuuri kirjeldus peatükkide kaupa. Samuti tutvustakse lühidalt lisasid, sh kaasapandud
materjalide sisu.
  \end{itemize}
  
  \newpage
  
  % Probleem
  \section{Probleemi püstitus}
  
  Süsteemi arhitektuur\\
  Nõuded lahendusele\\
  
  keskkondade arv (8 nendest productions): 26 + VM-s ja local machines\\
  Tomcat 8 serverite arv: 29 + VM-s ja local machines\\
  WebLogic 12 serverite arv: 2\\
  
  \newpage
  
  % Võimalikud lahendused
  \section{Võimalikud lahendused}
  
  \begin{tabular}{| l | c |}
    \hline
    Tarkvara & Hind aastas\\
    \hline
    Atlassian Bamboo & \$0*\\
    Chef & \$127/node\\
    Jenkins & \$0\\
    Ansible & \$14,000\\
    Bash & \$0\\
    \hline
  \end{tabular}

  
  \subsection{Atlassian Bamboo}
  \cite{bamboo}
  
  \subsection{Chef}
  \cite{chef}
  
  \subsection{Jenkins}
  \cite{jenkins}
  
  \subsection{Ansible Tower}
  \cite{ansible}

  \subsection{Bash skriptimine}
  \cite{bash}
  Bash skriptimiskeel - mis on selle eelised?
  
  \newpage
  
  % Lahendus
  \section{Lahendus}
  
  \subsection{Skripti arhitektuur ja tööpõhimõte}
  
  \subsection{Skripti paigaldus}
  
  \subsection{Skripti kasutus}
  
  \subsection{Dokumentatsioon}
  
  \newpage
  
  % Mõju projektile
  \section{Mõju projektile}
  
  \subsection{Statistika}
  
  \begin{tabular}{| l | l | l |}
    \hline
    Module Name & Script Time (sec) & Manual Time (sec)\\
    \hline
    admin & 27.245 & 80.45\\
    treatment & 61.101 & 112.71\\
    reception & 38.538 & 100.19\\
    \hline
  \end{tabular}
  
  \cite{time}

  
  \subsection{Kolleegide tagasiside}
  
  \subsubsection{Martin Rakver, Hosting Services and Application Manager}
  
  I have not been using the script very much, because I rarely update modules nowadays. But what I can say is that it is definately in the right place – automating activities that testers would need do to on daily basis each time when deploying a new module. Come to think about it, there are actually quite many activites related to new module deployment – would be interesting to compare the time it takes to deploy a module manually vs using script + calculate about how much time we as a team are saving daily,montly,yearly  (and get to do more important things with that time).
I also like that scalability has been considered when writing the script + many optional features have been implemented which can be used if one desires to do so.

  \subsubsection{Kalle Jagula, QA Specialist}
  
  I personally think that version-update script has significantly improved the speed of version update. Especially with the settings for bulk-update, which improves the speed of modules' version update and manageability of test/demo environments.

  
  % Lahenduse perspektiivid
  \section{Lahenduse perspektiivid}
  
  \newpage
  
  % Kokkuvõte
  \section*{Kokkuvõte}
  \label{kokkuvote}
  \addcontentsline{toc}{section}{\nameref{kokkuvote}}

Kokkuvõttes tuuakse selgelt välja töö põhilised saavutused. Ei tohi sisse tuua uusi väiteid, põhjendusi ja analüüse, mida töö põhiosas ei ole käsitletud. Mõistlik on viidata eesmärgile ja sõnastada kokkuvõte nii, et on näha eesmärgi saavutamine ning ka autori panus.\\

Kokkuvõttes võib lühidalt esile tuua töö edasiarendamise võimalikke teid ja perspektiive.\\

Kokkuvõtte tekstis on soovitav kasutada lihtmineviku umbisikulist tegumoodi, näiteks "Töös toodi välja …, kirjeldati …, leiti lahendus … ", aga ka olevik on vastuvõetav.

  \newpage
  
  % Kasutatud materjalid 
  \begin{thebibliography}{9}
    \label{kasutatud-materjalid}
    \addcontentsline{toc}{section}{\nameref{kasutatud-materjalid}}
  
    \bibitem{bash}
    \url{https://www.gnu.org/software/bash/manual/bashref.html}
    
    \bibitem{tomcat}
    \url{https://tomcat.apache.org/tomcat-8.0-doc/manager-howto.html}
  
    \bibitem{cron}
    \url{https://help.ubuntu.com/community/CronHowto}
  
    \bibitem{hg}
    \url{https://www.mercurial-scm.org/guide}
  
    \bibitem{centos}
    \url{https://wiki.centos.org/FrontPage}
  
    \bibitem{bamboo}
    \url{https://www.atlassian.com/software/bamboo}
  
    \bibitem{chef}
    \url{https://www.chef.io}
  
    \bibitem{jenkins}
    \url{https://jenkins-ci.org}
  
    \bibitem{ansible}
    \url{https://www.ansible.com}
    
    \bibitem{time}
    \url{http://pubs.opengroup.org/onlinepubs/9699919799/utilities/time.html}
 
  \end{thebibliography}

  \newpage
  
  % Lisad
  \section*{Lisad}
  \label{lisad}
  \addcontentsline{toc}{section}{\nameref{lisad}}
  
  \begin{itemize}
  
    \item Skripti manuaal: \url{https://github.com/iriiiina/bachelors-thesis/blob/master/manual/Confluence%20Manual.pdf}
    
    \item Skripti kood: \url{https://github.com/iriiiina/version-updater}
  
    \item BPMN diagramm ühe mooduli uuenduse kohta Tomcat-i peal (update-version-tomcat.sh): \url{https://github.com/iriiiina/bachelors-thesis/blob/master/thesis/pictures/BPMN-diagram-one-module-tomcat.png}
  
  \end{itemize}


\end{document}