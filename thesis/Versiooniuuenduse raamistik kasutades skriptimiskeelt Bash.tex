\documentclass[12pt]{report}
\usepackage[utf8]{inputenc}
\usepackage{nameref}
\usepackage{fancyhdr}
\renewcommand{\contentsname}{Sisukord}
\renewcommand{\thesection}{\arabic{section}}
% Page numbers to the right
\pagestyle{fancy}
\fancyhead{}
\fancyfoot{}
\fancyfoot[R]{\thepage}
\renewcommand{\headrulewidth}{0pt}

\title{Versiooniuuenduse raamistik kasutades skriptimiskeelt Bash}
\author{Irina Ivanova}
\date{January 2016}

\begin{document}
  \begin{titlepage}
    \begin{center}
      TARTU ÜLIKOOL\\
      MATEMAATIKA-INFORMAATIKATEADUSKOND\\
      Arvutiteaduse instituut\\
      Infotehnoloogia eriala
    \end{center}
      
    \vspace{5cm}
    
    \begin{center}  
      {\Large Irina Ivanova}
    \end{center}
    \begin{center}      
      {\huge Versiooniuuenduse raamistik}
    \end{center}
    \begin{center}
      {\huge kasutades skriptimiskeelt Bash}
    \end{center}
    \begin{center}
      {\large Bakalaureusetöö (6 EAP)}
    \end{center}
      
    \vspace{4cm}
    \hspace{4.5cm}
    Juhendajad: dotsent Helle Hein
    
    \hspace{7cm}
    Polina Morozova, MSc (Nortal AS)
      
    \vspace{2.5cm}
    \begin{center}
      TARTU 2016
    \end{center}
  \end{titlepage}

  \newpage
  
  Versiooniuuenduse raamistik kasutades skriptimiskeelt Bash\\
  
  Lühikokkuvõte:\\
  \vspace{2cm}

  Võtmesõnad:\\
  \vspace{2cm}

  Thesis title\\

  Abstract:\\
  \vspace{2cm}

  Keywords:
  
  \newpage
 
  \tableofcontents
  \fancypagestyle{plain}{%
    \renewcommand{\headrulewidth}{0pt}%
    \fancyhf{}%
    \fancyfoot[R]{\thepage}%
}
 
  \newpage
  
  % Sissejuhatus
  \section*{Sissejuhatus}
  \label{sissejuhatus}
  \addcontentsline{toc}{section}{\nameref{sissejuhatus}}
  
  \begin{itemize}
    \item teema valiku põhjendus (teema aktuaalsus ja uudsus)
    \item töö eesmärk
    \item  uuritav probleem (vajadusel püstitatud hüpotees(id), uurimisküsimus(ed), uuringu objekt)
    \item töö struktuuri kirjeldus peatükkide kaupa. Samuti tutvustakse lühidalt lisasid, sh kaasapandud
materjalide sisu.
  \end{itemize}
  
  \newpage
  
  % Probleem
  \section{Probleem}
  
  \subsection{Probleemi kirjeldus}
  
  \subsection{Süsteemi arhitektuur}
  
  \begin{tabular}{|l|c|c|}
    \hline
    Environment & Count & Web Server\\
    \hline
    Test-TC & 12 & Tomcat 8\\
    Live-TC & 8 & Tomcat 8\\
    Local & N* & Tomcat 8\\
    Test-WL & 4 & WebLogic 12\\
    \hline
  \end{tabular}

  
  \subsection{Nõuded lahendusele}
  
  \newpage
  
  % Võimalikud lahendused
  \section{Võimalikud lahendused}
  Bamboo, Chef, Jenkins, Ansible - mis on nende puudused?\\
  Bash skriptimiskeel - mis on selle eelised?
  
  \newpage
  
  % Lahendus
  \section{Lahendus}
  
  \subsection{Skripti arhitektuur ja tööpõhimõte}
  
  \subsection{Skripti paigaldus}
  
  \subsection{Skripti kasutus}
  
  \newpage
  
  % Kokkuvõte
  \section*{Kokkuvõte}
  \label{kokkuvote}
  \addcontentsline{toc}{section}{\nameref{kokkuvote}}

Kokkuvõttes tuuakse selgelt välja töö põhilised saavutused. Ei tohi sisse tuua uusi väiteid, põhjendusi ja analüüse, mida töö põhiosas ei ole käsitletud. Mõistlik on viidata eesmärgile ja sõnastada kokkuvõte nii, et on näha eesmärgi saavutamine ning ka autori panus.\\

Kokkuvõttes võib lühidalt esile tuua töö edasiarendamise võimalikke teid ja perspektiive.\\

Kokkuvõtte tekstis on soovitav kasutada lihtmineviku umbisikulist tegumoodi, näiteks "Töös toodi välja …, kirjeldati …, leiti lahendus … ", aga ka olevik on vastuvõetav.

  \newpage
  
  % Kasutatud materjalid
  \section*{Kasutatud materjalid}
  \label{kasutatud-materjalid}
  \addcontentsline{toc}{section}{\nameref{kasutatud-materjalid}}
  
  https://www.gnu.org/software/bash/manual/bashref.html
  
  https://tomcat.apache.org/tomcat-8.0-doc/manager-howto.html
  
  https://help.ubuntu.com/community/CronHowto
  
  https://www.mercurial-scm.org/guide
  
  https://wiki.centos.org/FrontPage

  \newpage
  
  % Lisad
  \section*{Lisad}
  \label{lisad}
  \addcontentsline{toc}{section}{\nameref{lisad}}

\end{document}