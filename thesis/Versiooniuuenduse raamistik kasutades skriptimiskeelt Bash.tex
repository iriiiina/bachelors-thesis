\documentclass[12pt]{report}
\usepackage[utf8]{inputenc}
\usepackage{nameref}
\usepackage{fancyhdr}
\usepackage{hyperref} 
\renewcommand{\contentsname}{Sisukord}
\renewcommand{\bibname}{Kasutatud materjalid}
\renewcommand{\thesection}{\arabic{section}}
% Page numbers to the right
\pagestyle{fancy}
\fancyhead{}
\fancyfoot{}
\fancyfoot[R]{\thepage}
\renewcommand{\headrulewidth}{0pt}

\title{Versiooniuuenduse raamistik kasutades skriptimiskeelt Bash}
\author{Irina Ivanova}
\date{January 2016}

\begin{document}
  \begin{titlepage}
    \begin{center}
      TARTU ÜLIKOOL\\
      MATEMAATIKA-INFORMAATIKATEADUSKOND\\
      Arvutiteaduse instituut\\
      Infotehnoloogia eriala
    \end{center}
      
    \vspace{5cm}
    
    \begin{center}  
      {\Large Irina Ivanova}
    \end{center}
    \begin{center}      
      {\huge Versiooniuuenduse raamistik}
    \end{center}
    \begin{center}
      {\huge kasutades skriptimiskeelt Bash}
    \end{center}
    \begin{center}
      {\large Bakalaureusetöö (6 EAP)}
    \end{center}
      
    \vspace{4cm}
    \hspace{4.5cm}
    Juhendajad: dotsent Helle Hein
    
    \hspace{7cm}
    Polina Morozova, MSc (Nortal AS)
      
    \vspace{2.5cm}
    \begin{center}
      TARTU 2016
    \end{center}
  \end{titlepage}

  \newpage
  
  \noindent{\textbf{Versiooniuuenduse raamistik kasutades skriptimiskeelt Bash}}\\
  
  \noindent{\textbf{Lühikokkuvõte:}}\\
  \vspace{2cm}
  
  \noindent{\textbf{Võtmesõnad:}}\\
  Bash skript, versiooni uuendus, Apache Tomcat, Java
  \vspace{2cm}

  \noindent{\textbf{Thesis title}}\\

  \noindent{\textbf{Abstract:}}\\
  \vspace{2cm}

  \noindent{\textbf{Keywords:}}\\
  Bash script, version update, Apache Tomcat, Java
  
  \newpage
 
  \tableofcontents
  \fancypagestyle{plain}{%
    \renewcommand{\headrulewidth}{0pt}%
    \fancyhf{}%
    \fancyfoot[R]{\thepage}%
}
 
  \newpage
  
  % Sissejuhatus
  \section*{Sissejuhatus}
  \label{sissejuhatus}
  \addcontentsline{toc}{section}{\nameref{sissejuhatus}}
  
  Käesolevas lõputöös lahendatakse rakenduslikku probleemi toimivas tarkvaraprojektis. Probleem seisneb selles, et projekti olemuse tõttu toote versiooni uuendamise protsess võtab kaua aega ja eeldab palju manuaalset sekkumist. See raiskab resursse ja suurendab inimliku vea tekkimise tõenäosust. Töö eesmärk on automatiseerida versiooniuuenduse protsessi konkreetses projektis, selleks, et lihtsustada antud protsessi, et säästa aega ja vähendada inimlike vigade arvu.
  
  Töö koosneb viiest osast. Esimeses osas püstitatakse probleem: kirjeldatakse projekti arhitektuuri, manuaalse uuendamise protsessi, probleeme, mis vana protsess toob ja nõudeid uuele lahendusele. Peamine väljakutse seisneb selles, et toode koosneb paarkümnest moodulitest (täpne arv sõltub kliendi nõuetest), mida on vaja paigaldada 31 keskkonna. See tähendab, et selleks, et paigaldada toote uut versiooni kõikidele keskkondadele on vaja teha umbes 620 uuendust (mitte arvestades testkeskkondade vaheuuendustega). Lisaks on projektis kasutusel kaks erinevat veebiserverit: Oracle WebLogic 12 ja Apache Tomcat 8. Uuenduse protsessi automatiseerimine peab võtma arvesse kõik need väljakutsed.
  
  Teises osas uuritakse võimalikke lahendusi: kas kasutada turul olemasolevat toodet (töös uuritakse neli toodet: Atlassian Bamboo, Chef, Jenkins ja Ansible Tower) või kirjutada ise tarkvara. Tuuakse välja iga lahenduse eeliseid ja puuduseid ning seletatakse miks langetati otsus Bash skripti kirjutamise kasuks. Valmistoodete peamised puudused on maksmus, võimetus hallata protsesse ja vajadusel parandada vigu, vajadus kulutada palju aega toode tundmiseks ja rakendamiseks.
  
  Kolmandas osas kirjeldatakse valitud lahendust: mis on skripti tööpõhimõtte ja arhitektuur ja kuidas seda saab paigaldada ja kasutada. Skripti põhifunktsionaalsus on mooduli faili alla laadimine, vana versiooni maha võtmine, uue versiooni paigaldamine, kolleegide teavitamine toimunud versiooniuuendusest. Lisaks oluliseks funktsionaalsuseks on erinevad kontrollid, mis vähendavad tõenäosust paigaldada vale versioon ja logimise funktsionaalsus, mis võimaldab kätte saada põhjalikku infot iga uuenduse kohta.
  
  Neljandas osas uuritakse kuidas antud lahendus mõjus projektile: võrreldakse versiooni uuendusele kulunud aega manuaalse ja automatiseeritud protsessi puhul ja lisatakse kolleegide tagasisidet uue raamistiku kasutamisele.
  
  Viimases osas tuuakse välja lahenduse perspektiive: kuidas loodud skripti saab edasi arendada ja kuidas saab seda teistes projektides rakendada.
  
  Töö tulemuseks on töötav, avatud lähtekoodiga, skript Bash keeles ja selle dokumentatsioon.
  
  \newpage
  
  % Probleem
  \section{Probleemi püstitus}
  
  Süsteemi arhitektuur\\
  Nõuded lahendusele\\
  
  keskkondade arv (8 nendest productions): 26 + VM-s ja local machines\\
  Tomcat 8 serverite arv: 29 + VM-s ja local machines\\
  WebLogic 12 serverite arv: 2\\
  
  \newpage
  
  % Võimalikud lahendused
  \section{Võimalikud lahendused}
  
  Uuenduse protsesside automatiseerimiseks ja haldamiseks on olemas palju valmistoodet. Kõige populaarsemad nendest on Atlassian Bamboo, Chef, Jenkins, Ansible Tower ja Puppet.
  
  Kõikide valmistoodete rakendamisel alati on olemas oma eelised ja puudused.
  
  Eelised:
  \begin{itemize}
    \item Ei ole vaja kulutada aega süsteemi kirjutamisele, mis tegelikult on juba leiutatud.
    \item Kuna rakendus on juba turul ja on populaarne, siis on suur tõenäosus, et see on kvaliteetne ja vastab kasutajate ootustele.
    \item On olemas kogukond tarkvara kasutajatest, kes vajadusel saavad aidata ja jagada enda kogemust.
  \end{itemize}
  
  Puudused:
  \begin{itemize}
    \item Suurem osa valmistoodetest on tasulised (Jenkins on erandiks) ja hind alati sõltub serverite, keskkondade ja moodulite arvust, mis vaadetud projekti korral on väga suur.
    \item Valmistoode õppimine, paigaldamine ja seadistamine võtab palju aega, eriti kui see pakkub lai valik funktsionaalsust.
    \item Projekt väga sõltub tootjast \-- vigade olemas olul ei saa olla kindel, et neid parandatakse lähimal ajal või parandatakse üldse; probleemide tekitamisel ei saa olla kindel, et tootja kasutajate tugi saab aidata.
    \item Projekt peab usaldama tootjat \-- usaldama, et valmistoode on turvaline ja stabiilne.
  \end{itemize}
  
  Need on ühtlased eelised ja puudused kõikide valmistoodete kohta. Järgmistes alampeatükides uuritakse iga toode eriomadusi.
  
  \subsection{Atlassian Bamboo}
  \cite{bamboo}
  
  Eelised:
  \begin{itemize}
    \item Kuna projektis kasutatakse Atlassian toodeid (JIRA, Confluence, Fisheye), siis Bamboo väga hästi integreerub nendega. On võimalik mudagvalt teha seoseid rakenduse koodi, muudatuste kirjelduse ja uuenduse vahel.
    \item Antud toodet kasutavad teised projektid firmas, mis annab võimalust kaasata inimesi, kellel on olemas teadmised ja kogemus Bamboo seadistamise kohta.
    \item Mugav kasutamine veebilehitseja kaudu.
  \end{itemize}
  
  Puudused:
  \begin{itemize}
    \item Seadistamine toimub moodulite kaupa (igal moodulil on oma Bamboo projekt) ja ühe mooduli seadistamine võtab palju aega. Firma teises projektis see aeg oli 1.5 kuud, mis antud projekti jaoks tähendaks ~20 * 1.5 = ~30 kuud.
    \item Üks uuendus võib võtta palju aega, sest Bamboo tööpõhimõte eeldab kolmanda serveri kasutamist uuenduse tööplaani teostamiseks ja selliste serverite arv on piiratud. See tähendab, et kui korraga soovitakse uuendada mitu keskkonda, siis võib tekkida ootusjärjekord.
  \end{itemize}
  
  \subsection{Chef}
  \cite{chef}
  
  Eelised:
  \begin{itemize}
    \item 
  \end{itemize}
  
  Puudused:
  \begin{itemize}
    \item 
  \end{itemize}

  
  \subsection{Jenkins}
  \cite{jenkins}
  
  Eelised:
  \begin{itemize}
    \item Tasuta.
    \item Avalik kood, mis tähendab, et projekt ei sõltu tootjast.
    \item Firmas on olemas spetsialistid, kes on kasutanud Jenkins-t ja saavad jagada teadmisi ja kogemust.
  \end{itemize}
  
  Puudused:
  \begin{itemize}
    \item 
  \end{itemize}
  
  \subsection{Ansible Tower}
  \cite{ansible}
  
  Eelised:
  \begin{itemize}
    \item 
  \end{itemize}
  
  Puudused:
  \begin{itemize}
    \item
  \end{itemize}
  
  \subsection{Puppet}
  \cite{puppet}
  
  Eelised:
  \begin{itemize}
    \item
  \end{itemize}
  
  Puudused:
  \begin{itemize}
    \item 
  \end{itemize}

  \subsection{Bash skriptimine}
  \cite{bash}
  Bash skriptimiskeel - mis on selle eelised?
  
  \newpage
  
  % Lahendus
  \section{Lahendus}
  
  \subsection{Skripti arhitektuur ja tööpõhimõte}
  
  \subsection{Skripti paigaldus}
  
  \subsection{Skripti kasutus}
  
  \subsection{Dokumentatsioon}
  
  \newpage
  
  % Mõju projektile
  \section{Mõju projektile}
  
  \subsection{Statistika}
  
  \begin{tabular}{| l | l | l |}
    \hline
    Module Name & Script Time (sec) & Manual Time (sec)\\
    \hline
    admin & 27.245 & 80.45\\
    treatment & 61.101 & 112.71\\
    reception & 38.538 & 100.19\\
    \hline
  \end{tabular}
  
  \cite{time}

  
  \subsection{Kolleegide tagasiside}
  
  \subsubsection{Martin Rakver, Hosting Services and Application Manager}
  
  I have not been using the script very much, because I rarely update modules nowadays. But what I can say is that it is definately in the right place – automating activities that testers would need do to on daily basis each time when deploying a new module. Come to think about it, there are actually quite many activites related to new module deployment – would be interesting to compare the time it takes to deploy a module manually vs using script + calculate about how much time we as a team are saving daily,montly,yearly  (and get to do more important things with that time).
I also like that scalability has been considered when writing the script + many optional features have been implemented which can be used if one desires to do so.

  \subsubsection{Kalle Jagula, QA Specialist}
  
  I personally think that version-update script has significantly improved the speed of version update. Especially with the settings for bulk-update, which improves the speed of modules' version update and manageability of test/demo environments.
  
  \subsubsection{Tanel Käär, System Architect}
  
  Version updater script has standardized the way our environments are updated, reducing learning curve for new people performing the task. Downside is that people are unaware, what happens in the background and can't handle simple errors in the process anymore.
  
  \subsubsection{Klaus-Eduard Runnel, Senior Programmer}
  
  I have mostly used the scripts to perform quick module and dependency updates in specific test environments when deploying quick fixes or debugging problems in the modules I'm responsible for. Deployment of artifacts is secondary task in such scenarios. The scripts allowed me to conveniently perform the updates while letting me concentrate on solving the primary problems and not leaving the environment containing all necessary information.

In some cases it was necessary to deploy custom-built experimental (unversioned) war-files. In these cases the scripts could not be used.
They would have been useful though if such experiments would have been commited to feature branches and published as feature branch artifacts. (As of today, we are better prepared for such circumstances.)

The scripts were also useful to deploy predefined sets of modules to temporary test environments living on reusable virtual machine images. It is important to note that a module update triggers launching a set of associated sql scripts to bring a database schema up to date. In that way the version update scripts let us automate most of a process of setting up updated environment from virtual machine images.

It would be helpful if the scripts were adapted to perform changes on remote machines and to perform changes on multiple environments simultaneously.

  
  % Lahenduse perspektiivid
  \section{Lahenduse perspektiivid}
  
  \newpage
  
  % Kokkuvõte
  \section*{Kokkuvõte}
  \label{kokkuvote}
  \addcontentsline{toc}{section}{\nameref{kokkuvote}}

Kokkuvõttes tuuakse selgelt välja töö põhilised saavutused. Ei tohi sisse tuua uusi väiteid, põhjendusi ja analüüse, mida töö põhiosas ei ole käsitletud. Mõistlik on viidata eesmärgile ja sõnastada kokkuvõte nii, et on näha eesmärgi saavutamine ning ka autori panus.\\

Kokkuvõttes võib lühidalt esile tuua töö edasiarendamise võimalikke teid ja perspektiive.\\

Kokkuvõtte tekstis on soovitav kasutada lihtmineviku umbisikulist tegumoodi, näiteks "Töös toodi välja …, kirjeldati …, leiti lahendus … ", aga ka olevik on vastuvõetav.

  \newpage
  
  % Kasutatud materjalid 
  \begin{thebibliography}{9}
    \label{kasutatud-materjalid}
    \addcontentsline{toc}{section}{\nameref{kasutatud-materjalid}}
  
    \bibitem{bash}
    \url{https://www.gnu.org/software/bash/manual/bashref.html}
    
    \bibitem{tomcat}
    \url{https://tomcat.apache.org/tomcat-8.0-doc/manager-howto.html}
  
    \bibitem{cron}
    \url{https://help.ubuntu.com/community/CronHowto}
  
    \bibitem{hg}
    \url{https://www.mercurial-scm.org/guide}
  
    \bibitem{centos}
    \url{https://wiki.centos.org/FrontPage}
  
    \bibitem{bamboo}
    \url{https://www.atlassian.com/software/bamboo}
  
    \bibitem{chef}
    \url{https://www.chef.io}
  
    \bibitem{jenkins}
    \url{https://jenkins-ci.org}
  
    \bibitem{ansible}
    \url{https://www.ansible.com}
    
    \bibitem{puppet}
    \url{https://puppet.com}
    
    \bibitem{time}
    \url{http://pubs.opengroup.org/onlinepubs/9699919799/utilities/time.html}
 
  \end{thebibliography}

  \newpage
  
  % Lisad
  \section*{Lisad}
  \label{lisad}
  \addcontentsline{toc}{section}{\nameref{lisad}}
  
  \begin{itemize}
  
    \item Skripti manuaal: \url{https://iriiiina.gitbooks.io/version-updater-manual/content/}
    
    \item Skripti kood: \url{https://github.com/iriiiina/version-updater}
  
    \item BPMN diagramm ühe mooduli uuenduse kohta Tomcat-i peal (update-version-tomcat.sh): \url{https://github.com/iriiiina/bachelors-thesis/blob/master/thesis/pictures/BPMN-diagram-one-module-tomcat.png}   
  
  \end{itemize}


\end{document}